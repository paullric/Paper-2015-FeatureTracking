\documentclass{article}

\usepackage{amsmath}
\usepackage{graphicx}
\usepackage{multicol}
\usepackage{color}
\usepackage{comment}

\oddsidemargin 0cm
\evensidemargin 0cm

\textwidth 16.5cm
\topmargin -2.0cm
\parindent 0cm
\textheight 24cm
\parskip 0.5cm

\usepackage{fancyhdr}
\pagestyle{fancy}
\fancyhf{}
%\fancyhead[L]{AOSS Reference Sheet}
%\fancyhead[CH]{test}
\fancyfoot[C]{Page \thepage}

\newcommand{\vb}{\mathbf}
\newcommand{\vg}{\boldsymbol}
\newcommand{\mat}{\mathsf}
\newcommand{\diff}[2]{\frac{d #1}{d #2}}
\newcommand{\diffsq}[2]{\frac{d^2 #1}{{d #2}^2}}
\newcommand{\pdiff}[2]{\frac{\partial #1}{\partial #2}}
\newcommand{\pdiffsq}[2]{\frac{\partial^2 #1}{{\partial #2}^2}}
\newcommand{\topic}{\textbf}
\newcommand{\arccot}{\mathrm{arccot}}
\newcommand{\arcsinh}{\mathrm{arcsinh}}
\newcommand{\arccosh}{\mathrm{arccosh}}
\newcommand{\arctanh}{\mathrm{arctanh}}

\begin{document}

{\Huge \textbf{TempestExtremes}}

\section{DetectCyclonesUnstructured}

\begin{verbatim}
Usage: DetectCyclonesUnstructured <parameter list>
Parameters:
  --in_data <string> [""] 
  --in_data_list <string> [""]
  --in_connect <string> [""] 
  --out <string> [""] 
  --out_file_list <string> [""]
  --searchbymin <string> [""] (default PSL)
  --searchbymax <string> [""] 
  --minlon <double> [0.000000] (degrees)
  --maxlon <double> [0.000000] (degrees)
  --minlat <double> [0.000000] (degrees)
  --maxlat <double> [0.000000] (degrees)
  --topofile <string> [""] 
  --maxtopoht <double> [0.000000] (m)
  --mergedist <double> [0.000000] (degrees)
  --closedcontourcmd <string> [""] [var,delta,dist,minmaxdist;...]
  --noclosedcontourcmd <string> [""] [var,delta,dist,minmaxdist;...]
  --thresholdcmd <string> [""] [var,op,value,dist;...]
  --outputcmd <string> [""] [var,op,dist;...]
  --timestride <integer> [1] 
  --regional <bool> [false] 
  --out_header <bool> [false] 
  --verbosity <integer> [0] 
\end{verbatim}

\begin{itemize}
\item[] \texttt{--in\_data <string>} \\ A list of input datafiles in NetCDF format, separated by semi-colons.
\item[] \texttt{--in\_data\_list <string>} \\ A file containing the \texttt{--in\_data} argument for a sequence of processing operations (one per line).
\item[] \texttt{--in\_connect <string>} \\ A connectivity file, which uses a vertex list to describe the graph structure of the input grid.  This parameter is not required if the data is on a latitude-longitude grid.
\item[] \texttt{--out <string>} \\ The output file containing the filtered list of candidates in plain text format.
\item[] \texttt{--out\_file\_list <string>} \\ A file containing the \texttt{--out} argument for a sequence of processing operations (one per line).
\item[] \texttt{--searchbymin <string>} \\ The input variable to use for initially selecting candidate points (defined as local minima).  By default this is ``PSL'', representing detection of surface pressure minima.  Only one of \texttt{searchbymin} and \texttt{searchbymax} may be set.
\item[] \texttt{--searchbymax <string>} \\ The input variable to use for initially selecting candidate points (defined as local maxima).  Only one of \texttt{searchbymin} and \texttt{searchbymax} may be set.
\item[] \texttt{--minlon <double>} \\ The minimum longitude for candidate points.
\item[] \texttt{--maxlon <double>} \\ The maximum longitude for candidate points.
\item[] \texttt{--minlat <double>} \\ The minimum latitude for candidate points.
\item[] \texttt{--maxlat <double>} \\ The maximum latitude for candidate points.
\item[] \texttt{--mergedist <double>} \\ Merge candidate points with distance (in degrees) shorter than the specified value.  Among two candidates within the merge distance, only the candidate with lowest \texttt{searchbymin} or highest \texttt{searchbymax} value will be retained. 
\item[] \texttt{--closedcontourcmd <cmd1>;<cmd2>;...} Eliminate candidates if they do not have a closed contour.  Closed contour commands are separated by a semi-colon.  Each closed contour command takes the form \texttt{var,delta,dist,minmaxdist}.  These arguments are as follows.
\begin{itemize}
\item[] \texttt{var <variable>}  The variable used for the contour search.
\item[] \texttt{dist <double>}  The great-circle distance (in degrees) from the pivot within which the closed contour criteria must be satisfied.
\item[] \texttt{delta <double>}  The amount by which the field must change from the pivot value.  If positive (negative) the field must increase (decrease) by this value along the contour.
\item[] \texttt{minmaxdist <double>}  The distance away from the candidate to search for the minima/maxima.  If \texttt{delta} is positive (negative), the pivot is a local minimum (maximum).
\end{itemize}
\item[] \texttt{--noclosedcontourcmd <cmd1>;<cmd2>;...} \\ As \texttt{closedcontourcmd}, except eliminates candidates if a closed contour is present.
\item[] \texttt{--thresholdcmd <cmd1>;<cmd2>;...}  Eliminate candidates that do not satisfy a threshold criteria (there must exist a point within a given distance of the candidate that satisfies a given equality or inequality).  Threshold commands are separated by a semi-colon.  Each threshold command takes the form \texttt{var,op,value,dist}.  These arguments are as follows.
\begin{itemize}
\item[] \texttt{var <variable>}  The variable used for the contour search.
\item[] \texttt{op <string>}  Operator that must be satisfied for threshold (options include \texttt{>}, \texttt{>=}, \texttt{<}, \texttt{<=}, \texttt{=}, \texttt{!=}).
\item[] \texttt{value <double>}  The value on the RHS of the comparison.
\item[] \texttt{dist <double>}  The great-circle-distance away from the candidate to search for a point that satisfies the threshold (in degrees).
\end{itemize}
\item[] \texttt{--outputcmd <cmd1>;<cmd2>;...}  Include additional columns in the output file.  Output commands take the form \texttt{var,op,dist}. These arguments are as follows.
\begin{itemize}
\item[] \texttt{var <variable>}  The variable used for the contour search.
\item[] \texttt{op <string>}  Operator that is applied over all points within the specified distance of the candidate (options include \texttt{max}, \texttt{min}, \texttt{avg}, \texttt{maxdist}, \texttt{mindist}).
\item[] \texttt{dist <double>}  The great-circle-distance away from the candidate wherein the operator is applied (in degrees).
\end{itemize}
\item[] \texttt{--timestride <integer>} \\ Only examine discrete times at the given stride (by default 1).
\item[] \texttt{--regional} \\ When a latitude-longitude grid is employed, do not assume longitudinal boundaries to be periodic.
\item[] \texttt{--out\_header} \\ Output a header describing the columns of the data file.
\item[] \texttt{--verbosity <integer>} \\ Set the verbosity level (default 0).
\end{itemize}

\subsection{Variable Specification} \label{sec:VariableSpecification}

Quantities of type \texttt{<variable>} include both NetCDF variables in the input file (for example, ``Z850'') and simple operations performed on those variables.  By default it is assumed that NetCDF variables are specified in the \texttt{.nc} file as
\begin{center}
\texttt{float Z850(time, lat, lon)} \quad or \quad \texttt{float Z850(time, ncol)}
\end{center} for structured latitude-longitude grids and unstructured grids, respectively.  If variables have no time variable, they have the related specification
\begin{center}
\texttt{float Z850(lat, lon)} \quad or \quad \texttt{float Z850(ncol)}
\end{center}  If variables include an additional dimension, for instance,
\begin{center}
\texttt{float Z(time, lev, lat, lon)} \quad or \quad \texttt{float Z(time, lev, ncol)}
\end{center} they may be specified on the command-line as $\texttt{Z(<lev>)}$, where the integer index \texttt{<lev>} corresponds to the first dimension (or the dimension after \texttt{time}, if present).  

Simple operators are also supported, including
\begin{itemize}
\item[] \texttt{\_ABS(<variable>)} Absolute value of a variable,
\item[] \texttt{\_AVG(<variable>, <variable>)} Pointwise average of variables,
\item[] \texttt{\_DIFF(<variable>, <variable>)} Pointwise difference of variables,
\item[] \texttt{\_F()}  Coriolis parameter,
\item[] \texttt{\_MEAN(<variable>, <distance>)} Spatial mean over a given radius,
\item[] \texttt{\_PLUS(<variable>, <variable>)} Pointwise sum of variables,
\item[] \texttt{\_VECMAG(<variable>, <variable>)} 2-component vector magnitude.
\end{itemize}  For instance, the following are valid examples of \texttt{<variable>} type,
\begin{center}
\texttt{\_MEAN(PSL,2.0)}, \quad \texttt{\_VECMAG(U850, V850)} \quad and \quad \texttt{\_DIFF(U(3),U(5))}.
\end{center}

\subsection{MPI Support} \label{sec:VariableSpecification}

The \texttt{DetectCyclonesUnstructured} executable supports parallelization via MPI when the \texttt{--in\_data\_list} argument is specified.  When enabled, the parallelization procedure simply distributes the processing operations evenly among available MPI threads.

\section{StitchNodes}

\begin{verbatim}
Usage: StitchNodes <parameter list>
Parameters:
  --in <string> [""] 
  --out <string> [""] 
  --format <string> ["no,i,j,lon,lat"] 
  --range <double> [5.000000] (degrees)
  --minlength <integer> [3] 
  --min_endpoint_dist <double> [0.000000] (degrees)
  --min_path_dist <double> [0.000000] (degrees)
  --maxgap <integer> [0] 
  --threshold <string> [""] [col,op,value,count;...]
  --timestride <integer> [1] 
  --out_format <string> ["std"] (std|visit)
\end{verbatim}

\begin{itemize}
\item[] \texttt{--in <string>} \\ The input file (a list of candidates from DetectCyclonesUnstructured).
\item[] \texttt{--out <string>} \\ The output file containing the filtered list of candidates in plain text format.
\item[] \texttt{--format <string>} \\ The structure of the columns of the input file.
\item[] \texttt{--range <double>} \\ The maximum distance between candidates along a path.
\item[] \texttt{--minlength <integer>} \\ The minimum length of a path (in terms of number of discrete times).
\item[] \texttt{--min\_endpoint\_dist <double>} \\ The minimum great-circle distance between the first candidate on a path and the last candidate (in degrees).
\item[] \texttt{--min\_path\_dist <double>} \\ The minimum path length, defined as the sum of all great-circle distances between candidate nodes (in degrees).
\item[] \texttt{--maxgap <integer>} \\ The largest gap (missing candidate nodes) along the path (in discrete time points).
\item[] \texttt{--threshold <cmd1>;<cmd2>;...} \\  Eliminate paths that do not satisfy a threshold criteria (a specified number of candidates along path must satisfy an equality or inequality).  Threshold commands are separated by a semi-colon.  Each threshold command takes the form \texttt{col,op,value,count}.  These arguments are as follows.
\begin{itemize}
\item[] \texttt{col <integer>}  The column in the input file to use in the threshold criteria.
\item[] \texttt{op <string>}  Operator used for comparison of column value (options include \texttt{>}, \texttt{>=}, \texttt{<}, \texttt{<=}, \texttt{=}, \texttt{!=}).
\item[] \texttt{value <double>}  The value on the right-hand-side of the operator. 
\item[] \texttt{count <integer>}  The minimum number of candidates along the path that must satisfy this criteria.
\end{itemize}
\item[] \texttt{--timestride <integer>} \\ Only examine discrete times at the given stride (by default 1).
\end{itemize}

\end{document}